\documentclass{article}
\usepackage{arxiv}
\usepackage[utf8]{inputenc} % allow utf-8 input
\usepackage[T1]{fontenc}    % use 8-bit T1 fonts
\usepackage{hyperref}       % hyperlinks
\usepackage{url}            % simple URL typesetting
\usepackage{booktabs}       % professional-quality tables
\usepackage{amsfonts}       % blackboard math symbols
\usepackage{nicefrac}       % compact symbols for 1/2, etc.
\usepackage{microtype}      % microtypography
\usepackage{lipsum}
\usepackage{graphicx}
\usepackage{subcaption}
\usepackage{amsmath}
\usepackage[numbers]{natbib}
\usepackage{float}
\usepackage[nottoc]{tocbibind}

\setcitestyle{authoryear,open={(},close={)}}

\title{A Spatial Analysis of Housing Prices and Supermarket Amenities in London}

\author{
  \large{Kengo Arao}\thanks{With heartfelt thanks to Dr David Candon for dedicated supervision and to Dr Andrei Potlogea for guidance in urban economics}      \thanks{Data and code for this paper are publicly available at \href{https://github.com/KengoA/spatial-analysis-london}{https://github.com/KengoA/spatial-analysis-london}} \\
  Exam Number: B059089 \\
  School of Economics\\
  The University of Edinburgh\\
  \texttt{kengoarao@outlook.com}
}

\begin{document}
\maketitle

\begin{abstract}
Integrating the bid rent theory by \citet{AlonsoWilliam1964Lalu}, \citet{MuthRichardF.1969Cah:}, and \citet{MillsEdwinSmith1972Sits} into the hedonic pricing model by \citet{Rosen1984}, I analyse the dynamics of housing prices in Greater London by constructing novel spatial features from Uber Movement and OpenStreetMap. I first estimate the commuting cost and find that the average travelling time on the road in the morning has a more substantial effect on housing prices than the geometric distance to the Central Business District, while the latter has a higher explanatory power overall across the metropolitan area. Heterogeneity in local amenities is introduced to investigate the effects of supermarket amenities, controlling for land use, food and beverage establishments, nature and tourism amenities, and access to education. Through cross-sectional analysis, I demonstrate that the presence of high-end supermarkets such as Waitrose and Marks and Spencer is positively associated with higher mean housing prices by 5\% in a given area, whereas discount stores such as Aldi and Lidl have an adverse effect of -8\%. Using planning permissions and retail floor spaces as instruments, I provide evidence of causal directions for these effects, especially for high-end supermarkets. A more granular approach at the individual store level is suggested to establish more precise estimates of causal effects.

\end{abstract}

% keywords can be removed
% \keywords{First keyword \and Second keyword \and More}
\begin{center}
    \textbf{Word Count}: 9,872
\end{center}

\newpage
\tableofcontents

\newpage
\section{Introduction} \label{section:intro}
In every metropolitan area, housing price has a strong influence over household consumption choices and hence important implications for quality of life. In light of the rapid urbanisation with 68\% of the world population expected to live in urban areas by 2050 \citep{UnitedNations201868UN}, the underlying mechanism of urban housing markets has been under heavy scrutiny in the fields of urban research and economic geography, as well as being a focal point of popular media outlets in the United Kingdom \citep{TheGuardian2019HousingMarket}. One particular example of dynamically changing urban landscapes is the penetration of German supermarkets into the British grocery market in recent years \citep{Davey2018AldiMarket}, with their low-price policies affecting local consumption behaviours. On the other end of the spectrum in terms of price levels, several newspaper articles popularised the term "the Waitrose effect" based on a report by \citet{LloydsBank2016LivingHome}, which argues that proximity to a high-end supermarket such as Waitrose implies a premium on housing prices upwards of 43,000 pound sterling \citep{TheIndependent2017WaitrosePounds}. The causal link between the two and the importance of neighbourhood characteristics on housing prices, nonetheless, remain to be understood with limited literature on European supermarket chains.

The objective of this paper is thus to investigate the effects of such supermarkets and other neighbourhood amenities on housing prices at the local level, using empirical models inspired by the bid rent theory by \citet{AlonsoWilliam1964Lalu}, \citet{MuthRichardF.1969Cah:}, and \citet{MillsEdwinSmith1972Sits} for the commuting cost to the Central Business District (CBD) and the hedonic pricing model by \citet{Rosen1984} for product differentiation with objective characteristics. Through the empirical application of these models, I provide novelty in two respects. The first is the construction of local amenity variables from geographic information provided by OpenStreetMap along with commuting time estimate from Uber Movement data, which are applicable beyond the scope of the Greater London Area that is the primary focus of this paper. The second is the causal analysis of supermarket amenities in relation to housing prices in London with instrumental variables, in which I demonstrate that the presence of high-end supermarkets such as Waitrose and Marks and Spencer cause higher housing prices, controlling for other local amenities and distance to the CBD.

This paper is structured as follows. Section \ref{section:lit} first provides the theoretical foundations of seminal models by \citet{AlonsoWilliam1964Lalu} and \citet{Rosen1984} and discusses the monocentricity assumption in the literature and in the context of London, as well as a review of research into local amenities that affect housing prices such as education and consumption accessibility. Section \ref{section:data} briefly describes data sources and their properties, followed by Section \ref{section:variables} where I illustrate data transformation and feature engineering methods based on existing research. Section \ref{section:model} describes regression models and results considering the theoretical framework and empirical literature, and provide causal analysis with instrumental variables for high-end and low-end supermarkets. Section \ref{section:discussion} discusses external validity of this paper's approach as well as limitations and future research, and Section \ref{section:conclusion} concludes.

\section{Conceptual Frameworks and Literature Review} \label{section:lit}
This section introduces two theories that are fundamental to the empirical models I develop in Section \ref{section:model}. Each theory is followed by a review of empirical research and their implications to this paper, and provides motivations for establishing empirical models in Section \ref{section:model}.

\subsection{Bid Rent Theory} \label{subsection:monocentric}
A starting point for understanding urban complexity is a geographic economic theory by \citet{AlonsoWilliam1964Lalu}, \citet{MuthRichardF.1969Cah:}, and \citet{MillsEdwinSmith1972Sits} called bid rent theory, which refers to the relationship between the distance from the the city centre and real estate prices. The basic intuition is that the housing price increases as one moves away from the hinterland and closer towards the CBD, and consumers face a trade-off between housing costs and commuting costs. A simple version of this model described below, the monocentric city model, underpins the empirical specifications throughout this paper and requires the following five assumptions, the first of which is relaxed in Section \ref{subsection:hedonic}.
\begin{itemize}
\setlength\itemsep{0.1em}
\item Residents are homogeneous and live in a featureless area with all the jobs located in the CBD
\item Each resident is a worker and commutes to a job in the CBD
\item Each worker receives an urban wage $w$ with inelastic labour supply
\item Commuting from distance $d$ entails a linear commuting cost $t(d) = \tau d$
\item Each worker consumes 1 unit of housing which is produced solely  with land
\end{itemize}
Let $p_H (d)$ denote the price of housing located at distance $d$ from the CBD, then a worker's preference can be written as a linear utility function:
\begin{center}
$u = w - p _ { H } ( d ) - t ( d ) = w - p _ { H } ( d ) - \tau d$
\end{center}
With an outside option in the hinterland $\overline{u}$, the spatial equilibrium requires
\begin{center}
$u = w - p _ { H } ( d ) - \tau d = \overline { u }$
\end{center}
Rearranging and partially differentiating $p_H (d)$ with respect to $d$ yields 
\begin{center}
$\frac { \partial p _ { H } ( d ) } { \partial d } = t ^ { \prime } ( d ) = - \tau \quad \forall d \leq \overline { d }$
\end{center}
Let $\overline{d}$ be the upper bound for distance and $\underline{r} > 0$ be the rent for alternative use of land. Then, the equilibrium rent can be written as:

\begin{center}
$p _ { H } ( d ) = \underline { r } + \int _ { d } ^ { \overline { d } } t ^ { \prime } ( \delta ) d \delta = \underline { r } + \tau ( \overline { d } - d )$
\end{center}
Since the alternative land rent $\underline{r}$ is difficult to estimate in practice, I flip the sign of the equation such that 
\begin{center}
$p _ { H } ( d ) = \overline { r } - \int _ { \underline { d } } ^ { d } t ^ { \prime } ( \delta ) d \delta = \overline { r } - \tau ( d - \underline { d } )$
\end{center}
where $\overline{r}$ is the rent in the CBD and $\underline{d}$ is the lower bound for distance, which is 0 at the CBD. This equation hence represents a function of the price of housing which is linearly decreasing in distance from the CBD $d$, which is simply
\begin{center}
$p _ { H } ( d ) = \overline { r } - \tau d$
\end{center}
This is the basis of the univariate regression models where I estimate the coefficient $\tau$ associated with two measures of distance between the CBD and a given area; orthodromic distance and commuting time cost approximated by Uber Movement Data \citep{UberMovement2019MovementCities}. $\overline{r}$ is a constant term which represents the rent in the CBD at distance $d = 0$.

\subsection{Validity of Monocentricity} \label{subsection:monocentricity}
The monocentric city model remains heavily influential in empirical investigations in urban economics, and location has been a dominant explanatory factor for housing prices and urban structures. However, the assumption of monocentric cities has been actively contested both in theoretical and empirical terms. \citet{RePEc:eee:regeco:v:12:y:1982:i:2:p:161-196} criticise the \textit{a priori} determination of the CBD and provide general equilibrium models where firms and agents endogenously decide on the equilibrium location of the employment centre.
From an empirical point of view, \citet{McMillen2001PolycentricMilwaukee.} highlights the rise of polycentric city structures in the United States and identifies employment subcenters in Milwaukee, where employment opportunities are scattered around across the metropolitan area with only 6.7\% of them located in the CBD. A more systematic study is conducted by \citet{Arribas-Bel2014The2010}, in which they analyse the time trend of high employment density in 359 U.S. metropolitan areas in 1990, 2000, and 2010, and find that employment centres are still concentrated in the central areas, with 57\% of all metropolitan areas exhibiting monocentric structures in 2010. It is noteworthy that there is a large variance in employment share figures in these studies, depending on the definition and scope of the CBD. In general, while there is an increasing number of empirical studies that showcase the polycentric urban structures, they appear to be exceptions rather than the norm in the United States.

In the context of London, \citet{EuropeanCommission1999ESDP:Union} describes London as a monocentric city given the strong transport connectivity to the periphery areas and concentration of economic activities in the city. On the other hand, there is limited yet meaningful evidence of the polycentric nature of London. Utilising the large scale data of Oyster card transactions in the London subway, \citet{Roth2011StructureFlows} investigate the individual intra-urban flows with the Tube network to identify several basins of attraction in the region, the largest of which is West End followed by the City of London area. Importantly, they also demonstrate the significance of the Docklands area as a destination where a number of financial firms are located. Similarly, \citet{Park2011SpatialAngeles} studies commuting flows based on labour market statistics and surveys, and identifies a cluster of employment in Westminster, Camden and the City areas. Despite the evidence of heterogeneity especially within the central London, the monocentric structure seems to be present at the scale of Greater London that encompasses its population of 8 million. Hence, this paper follows \citet{Park2011SpatialAngeles} and impose monocentricity assumption \citep{AlonsoWilliam1964Lalu} in examining London's urban structure. Identification of the CBD is explored in detail in Section \ref{subsection:CBD}.

\subsection{Hedonic Pricing Model and Heterogeneous Amenities} \label{subsection:hedonic} 
In the simple monocentric model, it was assumed that the city is a featureless plane without any urban amenities such as restaurants and cafes. This assumption is now relaxed by introducing heterogeneity in amenities where each subsection of the city has its own characteristics that residents assign specific values to. This is an extension of the hedonic pricing model by \citet{Rosen1984} where a product is valued by its internal characteristics and external factors that are both homogeneous and objective. In this model, the class of products offers a package of quantifiable characteristics represented by a vector 

\begin{center}
    $z = \left( z _ { 1 } , z _ { 2 } , \dots , z _ { n } \right)$
\end{center}

where $z_i$ is the $i$-th characteristic on a plane of features, which in our case corresponds to the amount of a certain amenity present in the neighbourhood. Importantly, market clearing prices $p(z)$ are determined by consumer tastes and production costs, which equalise across buyers and identify the demand structure:

\begin{center}
    $p ( z ) = p \left( z _ { 1 } , z _ { 2 } , \dots , z _ { n } \right)$
\end{center}

The existence of market equilibrium in pure competition and decomposition of products into attributes $z_i$ is the theoretical grounding of the multivariate regressions of this paper, where the first element is the distance from the CBD and $2, 3, ..., n$-th elements are features based on geographical distributions of urban amenities.

While the hedonic pricing model is often applied for a smaller unit of housing such as a house or a flat, I argue that Rosen's criterion for objectivity is better suited for characteristics of an entire neighbourhood, as some factors in a individual housing unit such as architectural styles and views from the window could entail heterogeneous preferences that are prone to subjectivity. The scope of a geographic unit and the definition of neighbourhood are introduced in Section \ref{subsection:boundaries}.

\subsection{Literature on Housing, Local Amenities, and Supermarkets} \label{subsection:lit:supermarkets}
Hedonic pricing model is a standard method in the study of housing markets and utilised by several key papers. Regarding urban amenities, \citet{Niu2016ModelingDemand} provide several proxies to measure accessibility to key economic activities in modelling urban housing prices in Beijing. In addition to the employment centre, they investigate the effects of education, consumption and medical accessibility, and calculate the household demand structure for housing prices. They demonstrate that the employment accessibility has by far the highest importance as an economic activity, followed by education, consumption, and medical services. While their results are limited in statistical significance, their findings motivate the construction of measures of consumption accessibility through the number of supermarkets, and education accessibility in the form of private schools in Section \ref{subsection:school}. In relation to education, \citet{DepartmentforEducation2017HouseMore} finds that home prices in England are higher in areas near top 10 percent schools in terms of standardised test performance, where houses have a premium of 8.0\% for primary schools and 6.4\% for secondary schools, holding everything constant.

Another important aspect of daily consumption accessibility that affects individuals universally across socioeconomic strata is supermarkets. Although peer-reviewed research into the effects of supermarkets on housing prices is quite limited and virtually nonexistent in the British context, there is a small literature on Walmart openings in the United States. With their low price policy and "big box" establishments, the largest grocery retailer in the U.S. has been attributed to both positive and negative economic effects \citep{TheEconomist2006OpeningEffect}. \citet{Basker2009TheIndustry} highlight the downward pressure on local inflation as neighbouring businesses and competitors are forced to lower their prices by 1 to 1.2\%, despite the net positive effect on job creation. Furthermore, \citet{Dube2007FirmSector} find that a new Walmart store reduces local retail revenue levels by 0.5 to 0.9\% at the county level. In regards to housing markets, \citet{Pope2015WhenAlways} suggest two channels through which a Walmart store affects property values. On one hand, increased accessibility to shopping in general increases housing prices, which is especially the case in more urban environments \citep{Emrath2002ExplainingPrices}. In contrast, they hypothesise that a Walmart store potentially decreases housing prices via negative externalities such as higher levels of congestion, pollution and local crime as well as loss of visual aesthetics. \citet{Linden2008EstimatesLaws}, for instance, demonstrates a causal impact of crime risk in the locality on housing prices. To test whether the benefits of accessibility outweigh the costs of negative externalities, they use hedonic pricing models with fixed effects on housing characteristics, and initially find that the Walmart openings are associated with lower housing prices by 2 to 4\% within 2.5 years of opening. Nonetheless, using a difference-in-difference specification, they show that houses around a new Walmart establishment were sold at a 2 to 3\% higher level than the baseline prices, accounting for fixed effects. Similarly, \citet{Slade2016Walmart} demonstrates that the development of a new Walmart store increases the land prices by 39\%, using difference-in-difference estimators. External validity of these results, however, is subject to multiple threats outside of the United States especially due to the spatial constraint in congested urban areas in Europe.

In the United Kingdom, \citet{LloydsBank2016LivingHome}'s report on the housing price premium of different supermarket chains in England and Wales was influential in the British media which led to the popularisation of the term "the Waitrose effect" \citep{TheIndependent2017WaitrosePounds}. Supermarkets in general are associated with positive added values in housing prices in varying degrees, with Waitrose and Sainsbury's having a strong premium of £38,666 and £27,939 on average, followed by Marks and Spencer (£27,1812), Tesco (£22,072), Iceland (£20,034), Coop (£17,904), and Morrisons (£10,558). The lower end of distribution includes discount supermarkets such as Asda, Lidl, and Aldi which have a marginal premium of £5,026, £3,926 and £1,333, respectively. This translates into a premium of 10\% for Waitrose and Sainsbury's, while those of Asda, Lidl and Aldi are limited to 1-2\%. However, their empirical strategy to arrive at these figures is deeply problematic, in that they compare the average housing price of a postcode area where a target supermarket is present and the average housing price of other areas in the same locality where it is not. This approach disregards heterogeneity in amenities in different areas and creates a downward bias to areas without any of the these supermarkets, which makes it difficult to isolate the effects of supermarket amenities especially in densely populated areas such as London where a neighbouring area could be vastly different in amenities. Considering the existing research on Walmart and a report by \citet{LloydsBank2016LivingHome}, I will investigate the effects of supermarket amenities in a more rigorous manner by controlling for other key measures of accessibility to economic activities within the framework of \citet{AlonsoWilliam1964Lalu} and \citet{Rosen1984}. Given the complexity of urban spaces, I compile and synthesise several datasets from different sources to encapsulate neighbourhood characteristics. To provide empirical evidence on the effects of these amenities, Section \ref{section:data} first describes the data sources and their properties, followed by Section \ref{section:variables} where I carefully construct variables from these datasets.


\section{Data} \label{section:data}
\subsection{Geographic Boundaries} \label{subsection:boundaries}
The geographic unit for this paper's analysis is based on output areas (OAs) which have been the basis for UK Census since 2001 and are designed to be homogeneous in terms of population sizes and household characteristics \citep{OfficeforNationalStatistics2019CensusGeography}. For compatibility with the housing prices data described below, an aggregate measure of Middle Layer Super Output Areas (MSOAs) is obtained from \citet{LondonDatastore2015LSOAAtlas} for the empirical analyses with the population ranging from 5000 to 15000 and the number of households from 2000 to 6000, in which there are 984 MSOAs with the average population of 8,346 in 2010 for Greater London.

\subsection{Housing Prices}
Urban rent in each neighbourhood is approximated by property prices registered with \citet{LandRegistry2019AverageLSOA} which records each sale of a housing unit at the individual dwelling level. This is aggregated for each MSOA at the end of the calendar year for mean and median prices. Table \ref{table:housing} contains descriptive statistics for mean housing prices in pounds for the year of 2018, which clearly shows that the prices are not normally distributed with the mean value approaching the 75\% percentile, as this is skewed by significantly high housing prices in the central London. Figure \ref{fig:price_distribution} maps the geographical distribution of mean housing prices visually both on raw and the log scale. As it better approximates a normal distribution, log-transformed mean housing prices are used for empirical analysis throughout this paper.

\begin{table}[t]
\caption{Descriptive Statistics for Mean Housing Prices (GBP)} 
  \label{table:housing} 
\begin{tabular}{lrrrrrrrr}
\toprule
{} &  count &       mean &        std &       min &       25\% &       50\% &       75\% &        max \\
\midrule
Value &  983.0 &  603715.49 &  387541.06 &  226536.0 &  391267.0 &  495010.0 &  673064.5 &  4416659.0 \\
\bottomrule
\end{tabular}
\end{table}

\begin{figure}[t]
  \centering
  \begin{subfigure}{.45\textwidth}
      \centering
      \includegraphics[width=.95\linewidth]{images/housing_raw_mean.png}
      \caption{Mean Housing Price}
      \label{fig:1(a)}
  \end{subfigure}
  \begin{subfigure}{.45\textwidth}
      \centering
      \includegraphics[width=.95\linewidth]{images/housing_log_mean.png}
      \caption{log(Mean Housing Price)}
      \label{fig:1(b)}
  \end{subfigure}
  \caption{Housing Price Distribution Across Greater London}
  \label{fig:price_distribution}
\end{figure}

\subsection{Distance}
Each MSOA is defined as a polygon with edges represented with geographic coordinates of latitudes and longitudes (i.e. Trafalgar Square in London is at 51.5080 N, 0.1281 W), which will be the basis for the geometric distance measure between each MSOA and the CBD in Section \ref{section:variables}. The second distance measure is morning commuting time on the road, which is extracted from \citet{UberMovement2019MovementCities} that provides anonymised travel data in the form of summary statistics such as mean travel times and standard deviation at a given hour of the day. This data covers major cities in North America, Europe, Australia and India where Uber usage is high. For Greater London, quarterly data is available since 2016, aggregated by weekdays and weekends.

\subsection{Geographic Information}
The central focus of this paper is the use of geographic features from OpenStreetMap, which is a collaborative mapping scheme launched in 2004. Map data is based on systematic ground surveys by registered contributors of more than 5 million as of 2019 \citep{OpenStreetMap2019Contributors}. The crucial feature of this data is the standardised tags that store metadata of the map objects, which come in the forms of points, lines, and polygons based on the geographic coordinate system. For instance, an object may consist of key-value pairs of:  \texttt{\{name:'Tesco', building:'supermarket', landuse:'retail'\}} and an ordered pair of longitude and latitude as a coordinate, or a vector of coordinates if it is a polygon. Importantly, \texttt{landuse}, \texttt{building} and \texttt{landuse} take standardised values to represent different entities. Other keys that are important are \texttt{type, natural, sport, leisure, historic, tourism, amenity} and \texttt{shop}. I exploit this structure in constructing variables for consumption amenities including supermarkets, restaurants and tourism amenities, in addition to land use and nature controls.

\subsection{Private School Data} \label{subsection:school}
As \citet{Niu2016ModelingDemand} highlighted, accessibility to good schools is an important factor in the household choices of housing consumption. A salient feature of the British educational system is the established presence of private schools that educates 7\% of all children in the nation \citep{Green2019WhatProblem}. \citet{Ndaji2016ACouncil} examine the difference in educational achievements between independent schools and state schools in the United Kingdom, and demonstrate that students in independent schools had better results by 0.64 GCSE grades, controlling for prior academic ability, socioeconomic disadvantages, and gender differences. This discrepancy is equivalent to additional two years of schooling by the age of 16. Hence, I posit that private school is a desirable education amenity that households with children are especially attracted to. As OpenStreetMap does not provide school type in detail, I compile an original dataset for private schools in Greater London in a programmatic manner by scraping a web page by \citet{IndependentSchoolsCouncil2019IndepnendentArea}, which provides a comprehensive list of private schools in London providing primary and secondary education. Addresses are then converted into longitudes and latitudes using the \texttt{geocoders.Nominatim} module from \texttt{geopy} library in Python, such that each school falls onto one of the MSOAs.

\section{Variable Construction} \label{section:variables}
\begin{figure}[t]
\begin{subfigure}{.5\textwidth}
  \centering
  \includegraphics[width=.9\linewidth]{images/cbd.png}
\caption{Distribution of CBD functions \citep{GreaterLondonAuthority2008LondonsImportance}}
  \label{fig:cbd}
\end{subfigure}%
\begin{subfigure}{.5\textwidth}
  \centering
  \includegraphics[width=.9\linewidth]{images/park2011.png}
  \caption{Commuting patterns in Greater London \citep{Park2011SpatialAngeles}}
  \label{fig:park2011}
\end{subfigure}
\caption{CBD functions in London}
\label{fig:2}
\end{figure}

\subsection{Defining the CBD}\label{subsection:CBD}
As Section \ref{subsection:monocentricity} underlined, the exogenous determination of the CBD is a strong assumption in empirical applications of the monocentric city model. In regard to this, \citet{GreaterLondonAuthority2008LondonsImportance} qualitatively identifies several sub-centres within inner London (Figure \ref{fig:cbd}). While different CBD functions span horizontally in the north of the River Thames, there appears to be an agglomeration in the West where several industries overlap such as retail, real estate, accounting and consulting, media and creative, and higher education. \citet{Park2011SpatialAngeles} demonstrates this trend by constructing a commuting network into the central area using labour market statistics that records origin-destination trips between local authorities. Figure \ref{fig:park2011} illustrates the commuting patterns in Greater London where the size of each vertex represents the number of commuting inflows with the direction indicated by an arrow. He identifies a cluster of employment in three boroughs of London as prominent destinations; Camden, Westminster, and the City. Considering these observations, this paper assumes  \texttt{Westminster 019} to be the CBD where London Victoria station is located, that handles a significant amount of traffic inflow as a central London railway terminus with connections to several Tube lines that lead to the city centre. While this does not necessarily reflect the original definition by \citet{AlonsoWilliam1964Lalu} that describes the CBD as an area of workplaces given the small size of the MSOA relative to Greater London, I posit that the commuting inflows and the existence of a transportation hub is a strong indicator of employment accessibility in a modern metropolitan space such as London.

\subsection{Distance Measures}
\subsubsection{Orthodromic Distance}
The first measure of distance is the distance between two points on a sphere; the centroid of a given area $C_k$ and the centroid of the CBD, $C_0$, which is then converted to miles for ease of interpretation. By construction, each MSOA is represented by a simple polygon where each vertex consists of a pair of longitude and latitude in the standard geographic coordinate system (Bolstad, 2016). Let $v_{i,j}$ be a vertex in a N-gon where $v_i$ is a longitude and $v_j$ is a latitude. Then the centroid for a given $k$ is expressed as 

\begin{center}
    $C_k = (C_{ki}, C_{kj})  = (\frac{1}{N} \Sigma_{i=1}^{N} v_i, \frac{1}{N} \Sigma_{j=1}^{N} v_j) $
\end{center}

where $C_{ki}$ and $C_{kj}$ correspond to the longitude and latitude for the centroid. Since the centroids are relatively close to each other as they are all located in the same region, the Haversine formula \citep{VanBrummelenGlen2017Hm:t} is used for numerical stability to calculate the distance $d$ between two points on a sphere\footnote{The Haversine formula is computationally better conditioned than the arc length formula when two points are close to each other on the sphere due to rounding errors}, $C_0$ and $C_k$, such that

\begin{center}
    $d =2 r \arcsin \left(\sqrt{\sin ^{2}\left(\frac{C_{0i}-C_{ki}}{2}\right)+\cos \left(C_{ki}\right) \cos \left(C_{0i}\right) \sin ^{2}\left(\frac{C_{0j}-C_{kj}}{2}\right)}\right)$
\end{center}

where $r$ is the radius of earth in a desired unit of length, which in this case miles with $r = 3956$. 

\begin{table}[t]
\centering
\caption{Descriptive Statistics for Distance Measures} 
  \label{table:distance} 
\begin{tabular}{lrrrrrrrr}
\toprule
{} &  count &     mean &     std &     min &      25\% &      50\% &      75\% &      max \\
\midrule
dist\_miles     &  714.0 &     6.58 &    3.08 &    0.57 &     4.21 &     6.37 &     8.83 &    16.12 \\
dist\_uber &  714.0 &  2151.82 &  815.78 &  183.22 &  1604.15 &  2165.57 &  2730.87 &  4294.00 \\
\bottomrule
\end{tabular}
\end{table}

\begin{figure}[t]
  \begin{subfigure}{.5\textwidth}
      \centering
      \includegraphics[width=\linewidth]{images/scatter_miles.png}
      \caption{log(Miles distance)}
      \label{fig:3(a)}
  \end{subfigure}
  \begin{subfigure}{.5\textwidth}
      \centering
      \includegraphics[width=\linewidth]{images/scatter_uber.png}
      \caption{log(Uber Distance in Average Minutes Travelled)}
      \label{fig:3(b)}
  \end{subfigure}
  \caption{Log-log Linearity Between Housing Price and Distance}
  \label{fig:log_log}
\end{figure}

\subsubsection{Uber Movement Distance}
To approximate the morning commuting cost, I aggregate four quarters of Uber Movement data \citep{UberMovement2019MovementCities} in 2018 from weekdays by calculating the pairwise average time of Uber rides between the CBD and a given MSOA during the AM peak period of 7:00-10:00. It is noteworthy that these two MSOAs do not necessarily have a origin-destination relationship, as Uber's GPS systems track all subsequent locations from the origin as it traverses through different areas to the final destination \citep{UberMovementUberMethodology}. The primary advantage of commuting time measure compared to the geometric distance is that it accounts for traffic bottlenecks that stem from terrain structures and road connections. One such instance that demonstrates this advantage with this relatively new dataset is a study by \citet{UberMovementTeam2018ExaminingClosure} that examines the effect of the closure of the London Tower Bridge. As there is a limited number of bridges across the River Thames, the three-month closure in late 2016 for structural maintenance led to significantly higher levels of traffic on both sides of the bridge, increasing travel times by up to 65\% for southbound trips and up to 30\% for northbound trips. In the framework of the monocentric city model, this variable partly addresses the problematic assumption of a featureless urban plane by introducing such traffic obstacles that have real-world consequences for accessibility.\\\\
Figure \ref{fig:price_distribution} demonstrated that the distribution of mean housing prices across MSOAs has a long right tail due to extremely high values around the CBD. Both distance measures constructed in this subsection are also slightly skewed as shown in Table \ref{table:distance} with high values towards the right end of the distribution. The units are miles for \texttt{dist\_miles} and seconds for \texttt{dist\_uber}. Figure \ref{fig:log_log} visualises the log-log linear relationships between these distance measures and mean housing prices. As they allow for interpretations in terms of percentage changes in explanatory and dependent variables, log-scaled values are used for the model specifications in Section \ref{subsection:specifications}.

\subsection{Local Amenities from OpenStreetMap} \label{subsection:variable}
This section describes the most important novelty of this paper, that is the construction of variables for neighbourhood amenities from OpenStreetMap to characterise modern urban structures. This is followed by introducing variables used in empirical models in Section \ref{section:model} with reference to existing research on their effects on housing prices. In the hedonic pricing model, local amenities correspond to the vector $z$ that characterises the product, which is the log average housing price of a given MSOA. Among tens of thousands of map objects in each area, I employ \texttt{points} and \texttt{polygons} to construct a variable for each amenity by counting the number of a given entity present in the MSOA. For instance, to construct a variable \texttt{tesco} for a supermarket amenity in the \texttt{City of London 001}, I count the number of occurrences of its standardised name \texttt{Tesco Express, Tesco, Tesco Metro} from each map object within the boundary of \texttt{City of London 001}. To the best of my knowledge, I am the first to utilise these OpenStreetMap objects to represent local amenities in analysing housing prices. As I demonstrate in Section \ref{section:model}, they serve as useful approximations for neighbourhood characteristics in terms of land use, nature amenities, and most importantly, consumption amenities that include supermarkets. However, this method is not without limitations. First, it implicitly assumes homogeneity within the same class such as \texttt{sainsburys} and \texttt{cafe}, while some of these establishments might have differing characteristics in reality. This is circumvented partially by excluding outlier entities such as Tesco Superstores in the \texttt{tesco} example. Second, there might be a difference between the time an entity, say, a Tesco Express store, opened and the time it was actually registered as a map object, which is potentially problematic for time series analysis. Potential limitations and their consequences are discussed in detail in Section \ref{subsection:limitations} reflecting on the empirical analysis of the hedonic pricing models. \\\\
As the main interest of this paper, I first construct supermarket amenities. The same supermarket chains are chosen as the report by \citet{LloydsBank2016LivingHome} to investigate their individual effects in Greater London. These are, with variable names in brackets, Waitrose (\texttt{waitrose}), Sainsbury's (\texttt{sainsburys}), Marks and Spencer (\texttt{ms}), Tesco (\texttt{tesco}), Iceland (\texttt{iceland}), Coop (\texttt{coop}), Morrisons (\texttt{morrisons}), Asda (\texttt{asda}), Lidl (\texttt{lidl}), and Aldi (\texttt{aldi}). To ensure enough variation and understand a larger trend, this is abstracted to three different categories based on price levels and premium reported in \citet{LloydsBank2016LivingHome}, where \texttt{supermarkets\_high} = (\texttt{waitrose, sainsburys, ms}), \texttt{supermarkets\_middle} = (\texttt{tesco, morrisons, coop, iceland}), and \texttt{supermarkets\_low} = (\texttt{asda, lidl, aldi}). Individual effects will also be reported. While OpenStreetMap has good coverage in terms of the number of establishments for most of these chains, some classes are disproportionately represented such as \texttt{coop} where there are only 13 stores registered across Greater London. These variables alone cannot explain the marginal effects of supermarket amenities as housing price is a function of a number of other characteristics, and empirical models would suffer severely from omitted variable bias. Therefore, the rest of this section outlines key control variables for housing prices that are used in empirical models in Section \ref{section:model} in addition to the distance measures described above.\\\\
To understand the dynamics of housing markets, it is essential to address agent heterogeneity with different preferences for locations. \citet{AlonsoWilliam1964Lalu} argues that offices and commerce are willing to bid the highest rent to secure access to the population in the inner city. On the other hand, manufacturing firms optimise their surplus by choosing to locate in the periphery as they require more land for factories. Figure \ref{fig:landuse} by \citet{OSullivan2011UrbanEconomics} illustrates this pattern by considering the interactions between firms and workers. With distance from the CBD on the $x$-axis and urban rent on the $y$-axis, workers base themselves around offices and factories, leading to a second spike in urban rent at a certain distance from the CBD. Since the primary motivation of this paper is to understand the implications for household consumption of housing, I consider these variations in firm locations as exogenous and control for the land use effects by introducing \texttt{retail, residential, industrial, construction}, and \texttt{commercial} variables using \texttt{landuse} key from OpenStreetMap.

\begin{figure}[t]
  \centering
  \includegraphics[width=0.8\linewidth]{images/land-use-patterns.png}
  \caption{Land Use Patterns \citep{OSullivan2011UrbanEconomics}}
  \label{fig:landuse}
\end{figure}

Another important aspect of London is its popularity among tourists as it attracts by far the largest number of overseas tourists in the United Kingdom \citep{OfficeforNationalStatistics2019OverseasSeries}. Studying 103 urban areas in Italy over the 1996-2007 period, \citet{Biagi2015DoesItaly} quantify the influence of tourism-related activities on local housing markets and construct a tourism index that summarise several amenities such as accommodation and museums, and find that a 1\% increase in tourism index is associated with a 0.2\% increase in housing prices on average as they raise the housing demand, controlling for other population and socioeconomic factors in the neighbourhood. A rather large magnitude of this coefficient is sensible considering that it is an aggregate index that captures several economic activities in areas where tourism is of particular importance. \citet{OfficeforNationalStatistics2013The2009} outlines tourism amenities in the United Kingdom and show that food and beverage serving services has the highest consumption amount in terms of pounds, followed by air passenger transport services, accommodation services, and cultural activities. Reflecting on this, I include tourism control variables under the standardised key \texttt{tourism} from OpenStreetMap that include \texttt{hotel, museum, artwork}, and \texttt{gallery}. Food and drink establishments are treated separately from tourism amenities as they also serve the local population, where pubs and restaurants are aggregated as \texttt{pub\_restaurant} due to their high correlation, with other outlets added as \texttt{cafe} and \texttt{fast\_food} based on the \texttt{amenity} key.\\\\
In addition, control variables for nature amenities are also considered in terms of green spaces. \citet{Gibbons2014TheApproach} examine the value of proximity to environmental amenities on property prices in England under a similar hedonic pricing framework at the post code level, while controlling for employment accessibility and labour market characteristics. They find that natural amenities command a premium in housing prices, where gardens, green space such as parks and playgrounds, and areas of water are each associated with a 1\% increase in housing prices for a 1\% increase in terms of land cover share in a given ward. These effects are partially controlled for by introducing \texttt{garden, pitch, park} and \texttt{playground} from \texttt{leisure} key. However, constructing a variable by counting polygons is not the most suitable measure for nature amenities, given that the variance within each class is likely to be quite large. For instance, Hyde Park should be differentiated from St. James Park due to their size while they might be categorised equally as \texttt{park}. Furthermore, fresh water amenities such as rivers would not be properly represented as they consist of a \texttt{line} or a \texttt{multiline} in OpenStreetMap with a vector of coordinates that spans across different areas. For these amenities, land cover shares or proximity to residential areas seem to be a superior method to consider their relationship with housing prices. Nonetheless, for simplicity and consistency, nature controls are added in the same fashion as other variables. Similarly, \texttt{nature\_reserve} and \texttt{golf\_course} are also added under this category as they pertain to the characteristic of large green spaces.\\\\
Finally, as studies by \citet{Niu2016ModelingDemand} and \citet{DepartmentforEducation2017HouseMore} showed in Section \ref{subsection:lit:supermarkets}, accessibility to education is crucial for households with children in primary and secondary education. The same variable construction policy is applied to the \texttt{private\_school} data compiled in Section \ref{subsection:school}. While this dataset does not contain the date of establishment, I consider the locations of these schools to be fixed across years given the low frequency of school relocations and openings, and assumed to be present in the 2018 data used for regressions. For comparative purposes, I also construct \texttt{school} variable from OpenStreetMap that includes all types of schools as its type is rarely specified in the \texttt{type} key, and test the hypothesis that \texttt{private\_school} is a better proxy for education accessibility that differentiates the quality of education. Higher education is also included with \texttt{university} variable, although this is unlikely to affect housing prices given the lack of generality in location choices of universities.

\section{Empirical Analysis} \label{section:model}
Given the two theoretical frameworks presented in Section \ref{section:lit} and variables constructed in Section \ref{section:variables}, this section first describes empirical strategies using cross-sectional analysis and present the preliminary results. The causal effects of supermarket amenities are then investigated more thoroughly using instrumental variables, with rationales behind the construction of instruments and their limitations.

\subsection{Model Specifications} \label{subsection:specifications}
First of all, the simplest case of the monocentric city model is tested using two distance measures separately. In Section \ref{subsection:monocentric}, I formulated that housing price is a linear function of distance to the CBD, that is, $P_H(d) = \overline{r} - \tau d$, where $\overline{r}$ denotes the rent in the CBD. Considering the log-log linear relationship between distance and housing prices, I run the two following OLS regressions using data from 2018, and estimate two different coefficients for the magnitude of commuting cost $\tau > 0$ for each distance; $\tau_1$ for geometric distance converted to miles and $\tau_2$ for Uber distance approximated by average morning commuting time, such that

\begin{gather*}
ln(P_i) = \alpha - \tau_1 ln(dist\_miles_i) + \epsilon_i \\
ln(P_i) = \alpha - \tau_2 ln(dist\_uber_i) + \epsilon_i
\end{gather*}
\\
where $P_i$ is the mean housing price for a given MSOA $i$, and $\alpha$ is the coefficient for the constant term that corresponds to the rent $\overline{r}$ in the CBD in Section \ref{subsection:monocentric}. Following the analysis of these two models, I proceed by adopting one distance measure and setting up hedonic pricing specifications to investigate the effects of local amenities introduced in Section \ref{section:variables}. The key explanatory variables here are a vector of supermarkets $\boldsymbol{x}$, which contains three variables aggregated by premium levels reported in \citet{LloydsBank2016LivingHome}. Individual effects are also reported by supermarket chains. Adding the local amenities data extracted from OpenStreetMap data and private school data, the hedonic pricing model is specified as

\begin{gather*}
ln(P_i) = \alpha - \tau ln(dist_i) + \boldsymbol{x_i^{T}} \beta + \boldsymbol{land_i^{T}}\lambda + \boldsymbol{nature_i^{T}} \gamma + \boldsymbol{tourism_i^{T}} \theta + \boldsymbol{food_i^{T}} \phi + \boldsymbol{edu_i^{T}} \mu + \epsilon_i \\
\end{gather*}
such that
\begin{gather*}
\boldsymbol{x} = (supermarkets\_high, supermarkets\_middle, supermarkets\_low) \\
\boldsymbol{land} = (residential, retail, industrial, construction, commercial) \\
\boldsymbol{nature} = (garden, pitch, park, playground, nature\_reserve, golf\_course) \\
\boldsymbol{tourism} = (hotel, attraction, museum, art) \\
\boldsymbol{food} = (pub\_restaurant, cafe, fast\_food) \\ 
\boldsymbol{edu} = (school, private\_school, university) \\ 
\end{gather*}
with each of ($\beta, \lambda, \gamma, \theta, \phi, \mu$) representing a vector of associated coefficients for amenities within the category.  

\subsection{Results}
\subsubsection{Monocentric City Model}
% Table created by stargazer v.5.2.2 by Marek Hlavac, Harvard University. E-mail: hlavac at fas.harvard.edu
% Date and time: Wed, Mar 27, 2019 - 03:21:58
\begin{table}[t]\ \centering 
  \caption{Commuting cost comparison} 
  \label{table:monocentric} 
\small 
\begin{tabular}{@{\extracolsep{-10pt}}lcc} 
\\[-1.8ex]\hline 
\hline \\[-1.8ex] 
 & \multicolumn{2}{c}{\textit{Dependent variable:}} \\ 
\cline{2-3} 
\\[-1.8ex] & \multicolumn{2}{c}{log\_price} \\ 
\\[-1.8ex] & (1) & (2)\\ 
\hline \\[-1.8ex] 
 dist\_miles\_log & $-$0.583$^{***}$ &  \\ 
  & (0.021) &  \\ 
  & & \\ 
 dist\_uber\_log &  & $-$0.649$^{***}$ \\ 
  &  & (0.025) \\ 
  & & \\ 
 Constant & 14.299$^{***}$ & 18.197$^{***}$ \\ 
  & (0.038) & (0.193) \\ 
  & & \\ 
\hline \\[-1.8ex] 
Observations & 714 & 714 \\ 
R$^{2}$ & 0.526 & 0.478 \\ 
Adjusted R$^{2}$ & 0.525 & 0.477 \\ 
\hline 
\hline \\[-1.8ex] 
\textit{Note:}  & \multicolumn{2}{r}{$^{*}$p$<$0.1; $^{**}$p$<$0.05; $^{***}$p$<$0.01} \\ 
\end{tabular} 
\end{table}
Table \ref{table:monocentric} compares two distance measures as a proxy for employment accessibility in the monocentric city model, using univariate OLS regressions. As there is a nontrivial number of missing values for the pairwise Uber distance between the CBD and some areas with $N$ = 714 compared to 951 MSOAs with valid housing price data, the corresponding MSOAs are dropped from model (1) for distance in miles such that the estimated coefficients are directly comparable. Constants 14.299 in model (1) and 18.197 in model (2) are estimated housing prices in the CBD, $\overline{r}$, when the distance is equal to 0. Both models yield negative coefficients for distance that are statistically significant at the 1\% level, which is in line with the theoretical result from Section \ref{subsection:monocentric} where housing prices decrease as the distance from the CBD increases. To examine the difference between the two estimated coefficients $\tau_1$ and $\tau_2$ in two different models, I follow \citet{Clogg1995StatisticalModels.} and calculate that $Z = \frac{\tau_1 - \tau_2}{\sqrt{(SE\tau_1)^2 + (SE\tau_2)^2}}  = \frac{-0.583 - (-0.649)}{\sqrt{0.021)^2 + (0.025)^2}} = 2.021$, with $P(Z > 2.021) = 0.022$  which is statistically significant at the 5\% level. A log-log linear relationship allows interpretations in percentage terms, where a 1\% increase in distance in miles is associated with a 0.583\% lower housing price, and a 1\% increase in morning commuting time is associated with a 0.649\% lower housing price.  A steeper curve with Uber distance implies that commuting cost in terms of travel times is associated with a deeper discount to the housing prices compared to the discount associated with the geometric distance. While it appears sensible that actual commuting time has a higher impact compared to a rather raw distance measure of miles, model (1) explains a higher proportion of variance with Adjusted-$R^2$ of 0.525, in contrast to 0.477 for model (2). The higher explanatory power of model (1) potentially stems from the fact that morning travel time on the road approximated by Uber Movement data does not consider different modes of transportation that commuters opt to use. Well-developed underground and railway networks with good access to the periphery areas, for instance, are a differentiating characteristic of London compared to large American cities, which consist of combined 23\% of all journeys in Greater London by residents \citep{TransportforLondon2016TravelLondon}. The congestion charge in the inner city might also be a reason why people might shy away from commuting on the road \citep{TransportforLondon2019CongestionCharge}. Moving forward to the hedonic pricing model with additional control variables and supermarket amenities, I employ \texttt{dist\_miles\_log} as the main proxy for distance given the higher adjusted-$R^2$ and the completeness of the dataset for the geometric distance measure.

\subsubsection{Hedonic Pricing Model} \label{subsubsection:result:hedonic}
% Table created by stargazer v.5.2.2 by Marek Hlavac, Harvard University. E-mail: hlavac at fas.harvard.edu
% Date and time: Wed, Mar 27, 2019 - 03:45:07
\begin{table}[t] \centering 
  \caption{Cross-sectional OLS} 
  \label{table:hedonic} 
\small 
\begin{tabular}{@{\extracolsep{-10pt}}lccccccc} 
\\[-1.8ex]\hline 
\hline \\[-1.8ex] 
 & \multicolumn{7}{c}{\textit{Dependent variable:}} \\ 
\cline{2-8} 
\\[-1.8ex] & \multicolumn{7}{c}{log\_price} \\ 
\\[-1.8ex] & (1) & (2) & (3) & (4) & (5) & (6) & (7)\\ 
\hline \\[-1.8ex] 
 dist\_miles\_log & $-$0.513$^{***}$ & $-$0.494$^{***}$ & $-$0.469$^{***}$ & $-$0.486$^{***}$ & $-$0.470$^{***}$ & $-$0.469$^{***}$ & $-$0.440$^{***}$ \\ 
  & (0.015) & (0.015) & (0.015) & (0.015) & (0.015) & (0.015) & (0.015) \\ 
  & & & & & & & \\ 
 supermarkets\_high &  & 0.099$^{***}$ & 0.090$^{***}$ & 0.084$^{***}$ & 0.067$^{***}$ & 0.066$^{***}$ & 0.050$^{***}$ \\ 
  &  & (0.020) & (0.020) & (0.019) & (0.019) & (0.019) & (0.018) \\ 
  & & & & & & & \\ 
 supermarkets\_middle &  & $-$0.024$^{*}$ & $-$0.027$^{**}$ & $-$0.018 & $-$0.019 & $-$0.022$^{*}$ & $-$0.023$^{*}$ \\ 
  &  & (0.012) & (0.013) & (0.012) & (0.013) & (0.013) & (0.012) \\ 
  & & & & & & & \\ 
 supermarkets\_low &  & $-$0.143$^{***}$ & $-$0.116$^{***}$ & $-$0.112$^{***}$ & $-$0.095$^{***}$ & $-$0.093$^{***}$ & $-$0.088$^{***}$ \\ 
  &  & (0.024) & (0.024) & (0.023) & (0.022) & (0.022) & (0.022) \\ 
  & & & & & & & \\ 
 pub\_restaurant &  &  &  &  & 0.010$^{***}$ & 0.009$^{***}$ & 0.010$^{***}$ \\ 
  &  &  &  &  & (0.003) & (0.003) & (0.003) \\ 
  & & & & & & & \\ 
 cafe &  &  &  &  & 0.005 & 0.005 & 0.003 \\ 
  &  &  &  &  & (0.006) & (0.006) & (0.005) \\ 
  & & & & & & & \\ 
 fast\_food &  &  &  &  & $-$0.026$^{***}$ & $-$0.026$^{***}$ & $-$0.025$^{***}$ \\ 
  &  &  &  &  & (0.005) & (0.005) & (0.005) \\ 
  & & & & & & & \\ 
 school &  &  &  &  &  & 0.010$^{*}$ &  \\ 
  &  &  &  &  &  & (0.005) &  \\ 
  & & & & & & & \\ 
 private\_school &  &  &  &  &  &  & 0.124$^{***}$ \\ 
  &  &  &  &  &  &  & (0.014) \\ 
  & & & & & & & \\ 
 university &  &  &  &  &  & $-$0.0002 & $-$0.002 \\ 
  &  &  &  &  &  & (0.006) & (0.006) \\ 
  & & & & & & & \\ 
 Controls: Land Use &  &  & YES  & YES  & YES &  YES & YES \\ 
  &  &  &  &  &  &  & \\ 
 Controls: Nature &  &  &   & YES  & YES & YES & YES \\ 
  &  &  &  &  &  &  &  \\ 
 Controls: Tourism &  &  &  &  & YES & YES & YES  \\ 
  &  &  &  &  &  &  & \\ 
 Constant & 14.330$^{***}$ & 14.300$^{***}$ & 14.230$^{***}$ & 14.260$^{***}$ & 14.220$^{***}$ & 14.200$^{***}$ & 14.140$^{***}$ \\ 
  & (0.035) & (0.036) & (0.037) & (0.037) & (0.038) & (0.039) & (0.037) \\ 
  & & & & & & & \\ 
\hline \\[-1.8ex] 
Observations & 951 & 951 & 951 & 951 & 951 & 951 & 951 \\ 
R$^{2}$ & 0.543 & 0.570 & 0.596 & 0.631 & 0.652 & 0.653 & 0.680 \\ 
Adjusted R$^{2}$ & 0.543 & 0.568 & 0.592 & 0.625 & 0.644 & 0.644 & 0.671 \\ 
\hline 
\hline \\[-1.8ex] 
\textit{Note:}  & \multicolumn{7}{r}{$^{*}$p$<$0.1; $^{**}$p$<$0.05; $^{***}$p$<$0.01} \\ 
\end{tabular}
\end{table}

\begin{table}[h]
  \caption{Individual Supermarket Effects} 
  \label{table:ols:individual} 
\small
\begin{tabular}{@{\extracolsep{1pt}} ccccccccccc} 
\\[-1.8ex]\hline 
\hline \\[-1.8ex] 
 & waitrose & ms & sainsburys & tesco & iceland & morrisons & coop & lidl & asda & aldi \\ 
\hline \\[-1.8ex]
Count & 124 & 76 & 94 & 526 & 115 & 64 & 13 & 103 & 29 & 50 \\ 
Estimate & $0.052$ & $0.060$ & $0.032$ & $-0.011$ & $-0.056$ & $-0.028$ & $-0.116$ & $-0.102$ & $-0.061$ & $-0.085$ \\ 
Std. Error & $0.030$ & $0.043$ & $0.038$ & $0.015$ & $0.031$ & $0.033$ & $0.085$ & $0.031$ & $0.043$ & $0.055$ \\ 
t-value & $1.705$ & $1.408$ & $0.826$ & $-0.741$ & $-1.812$ & $-0.848$ & $-1.368$ & $-3.281$ & $-1.436$ & $-1.536$ \\ 
Pr(\textgreater \textbar t\textbar ) & $0.088$ & $0.159$ & $0.409$ & $0.459$ & $0.070$ & $0.397$ & $0.172$ & $0.001$ & $0.151$ & $0.125$ \\ 
\hline \\[-1.8ex] 
\end{tabular} 
\end{table} 
In Section \ref{subsection:monocentricity}, I posit that the hedonic pricing model extends the simple monocentric city model by incorporating heterogeneous amenities present in each neighbourhood. Table \ref{table:hedonic} follows this empirically with regression models where control variables are gradually added by different categories specified in Section \ref{subsection:specifications}. Model (1) corresponds to the univariate regression with distance in miles from the previous analysis of the monocentric city model. Model (2) first introduces supermarket variables at the aggregate level with three different price levels, and yield a preliminary result that an increase in the number of supermarket establishments in the MSOA is associated with a 9.9\% increase in housing prices for high-end supermarkets (Waitrose, Sainsbury's, and Marks and Spencer), a 2.4\% decrease for mid-level chains (Tesco, Morrisons, Coop, and Iceland), and a rather strong 14.3\% decrease for low-end discount stores (Lidl, Asda and Aldi). This model, however, is subject to omitted variable bias, and models (3)-(7) attempt to correct this bias by introducing control variables that are shown in the literature to affect housing prices through different channels. In this section, I first highlight the key control variables from OLS results and offer intuitive interpretations with reference to previous literature. This is followed by revisiting coefficients for supermarket variables in model (7), along with inspection into the individual effects by each supermarket chain.\\\\
In terms of explained variance, the inclusion of control variables contributes by a similar margin for each subsequent model in Table \ref{table:hedonic} with adjusted-$R^2$ increasing from 0.568 for model (2) to 0.592 for model (3) with land use controls, to 0.625 for model (4) with nature amenity controls, and to 0.644 for model (5) with tourism amenity controls and food and drink establishments. The complete regression table for model (7) is available in Appendix as Table \ref{table:complete}. Signs of coefficients are mostly in line with the existing studies introduced in Section \ref{section:variables} as part of variable construction. For instance, a tourism amenity \texttt{attraction} has a positive coefficient of 0.018 in model (7) that is statistically significant at the 10\% level, which implies that an increase in the number of tourist attractions is correlated with a 1.8\% higher housing prices. While it is not directly comparable due to variable construction, the modest positive effect is somewhat consistent with \citet{Biagi2015DoesItaly}'s result from Italian urban areas where a 1\% increase in tourism index is associated with 0.2\% higher housing prices in the neighbourhood. Interestingly, while \texttt{pub\_restaurant} has a positive coefficient of 0.010, \texttt{fast\_food} is associated negatively with the the housing price of -0.025, both statistically significant at the 1\% level. I argue that this somewhat strong implication for fast food is most likely to be driven by reverse causality. Given that fast food establishments are predominantly owned by large international chains such as McDonald's or Kentucky Fried Chicken, their location choices are expected to be more strategic. In fact, fast food representatives from American food corporations such as Wendy's and Chick-fil-A said in an interview that their restaurant location decisions are vitally influenced by geographic and demographic information \citep{Ungerleider2014HowLocation}. Combined with the fact that there is extensive literature on the relationship between low socioeconomic status and fast food consumption \citep{Appelhans2012SocioeconomicPurchases.}, it appears sensible that the negative coefficient for fast food outlets  is a consequence of their customer demographic that were already present.\\\\
Land use variable \texttt{industrial} and \texttt{commercial}, on the other hand, have negative coefficients of -0.022 and -0.018 both statistically significant at the 1\% level. It should be noted that \texttt{commercial} refers to large-scale services such as laboratories, business parks, warehouses and car repair stations, which is distinguished from \texttt{retail} tag which primarily contains food and drink establishments \citep{OpenStreetMapTag:landuse=commercial}. While the land use variation illustrated in Figure \ref{fig:landuse} qualitatively shows that the housing price increases nearby manufacturing and commercial centres, these variables appear to predict otherwise. One potential explanation for this is the decreasing number of employment opportunities in the manufacturing industry which has gone through automation for the past few decades, with employment in the industrial sector falling from 6.7 million in 1978 to 2.6 million in 2013 \citep{Jackson2018ManufacturingYears}. With lower demand from workers as employment centres, factories and commercial centres might be more strongly linked to negative externalities in the form of air pollution, noise, congestion, and undermined aesthetic attributes described in \citet{Orford1998ValuingMarket}, that likely decrease housing prices in the neighbourhood.\\\\
Nature controls highlight interesting suburban amenities and potential problems in the variable construction method. Table \ref{table:complete} shows that \texttt{golf\_course} has a strong positive coefficient of 0.144 that is statistically significant at the 1\% level. This variable identifies suburban amenities associated with affluent areas where gold courses are located. On the other hand, \texttt{parks} and \texttt{playgrounds} are shown to be negatively associated with housing prices, with coefficients of -0.003 (statistically significant at the 10\% level) and -0.022 (statistically significant at the 1\% level), respectively. This is in contrast to findings by \citet{Gibbons2014TheApproach} where green spaces provide value in terms of housing prices. As suggested in \ref{subsection:variable}, this partially stems from the heterogeneity in nature amenities in terms of size, that distorts their representation with the counting method used throughout this paper. I argue that this is also because of the complexity of nature amenities. For instance, a playground in the city centre likely provides more utility to residents than a playground in sparsely populated suburbs. Another factor to consider is that there is spatial competition between certain entities and amenities within a given area. While the hedonic pricing model implicitly assumes that the product can be decomposed into independent qualities that customers assign values to, this is often not the case for housing prices due to spatial constraint. This issue is especially severe in cases where a certain entity requires more space than others, such as parks and playgrounds, as it might come with the opportunity cost of building other amenities. While there are studies such as \citet{Niu2018DetailedWuhan} that investigate the spatial distribution of parks and their accessibility and efficiency as an urban amenity, this is beyond the scope of this paper and discussed as part of future work in Section \ref{subsection:limitations}.\\\\
Model (6) in  Table \ref{table:hedonic} introduces education variables from OpenStreetMap, only to add limited explanatory power to model (5) with a coefficient of 0.010 for \texttt{school}, significant at the 10\% level. On the other hand, an alternative variable for education \texttt{private\_school} compiled in Section \ref{subsection:school} has a much more significant effect on housing prices with a coefficient of 0.124 in model (7), statistically significant at the 1\% level. This verifies findings by \citet{Niu2016ModelingDemand} on the importance of education, and the housing premium of 12\% is more or less comparable to figures by Department for Education (2017) from their study in England with the premium of 8\% for primary schools in the top 10\% and 6.4\% for secondary schools. This is another example of a strongly positive amenity identified not only in the central but also in the periphery of London, contributing to increased explained variance in terms of adjusted-$R^2$ from 0.644 in model (5) to 0.671 in model (7). This result reinforces the result by \citet{Ndaji2016ACouncil} where independent schools are shown to have better educational outcomes in contrast to state schools, in the sense that this difference surfaces in the housing market in the form of housing premium.\\\\
After controlling for these amenities that influence housing prices, coefficients associated with supermarkets are lower in absolute values on both ends of the price spectrum, with 0.050 for high-end supermarkets and -0.088 for low-end supermarkets in model (7) compared to 0.099 and -0.143 for model (1), while maintaining the statistical significance at the 1\% level. It is noteworthy that adding private schools and food and beverage establishments has a rather large impact on the magnitude of these coefficients, and implies that these controls somewhat mitigate the omitted variable bias. Mid-tier supermarkets, on the other hand, appear to show a mixed result in terms of statistical significance, with the coefficient of -0.023 in model (7) is on a par with -0.024 in model (1). To investigate the individual effects, another regression model is run without aggregation of supermarkets by price levels using the same control variables as model (7). Table \ref{table:ols:individual} shows individual supermarket effects, with the number of stores for each chain across the entirety of Greater London shown on the first row. At point estimates, only supermarkets categorised in \texttt{supermarkets\_high} seem to command a premium with a positive coefficient, though a high standard error of 0.038 for \texttt{sainsburys} cannot leave out the possibility that the effect is zero. Lidl is statistically significant at the 1\% level and has a strong negative coefficient of -0.102, which implies that an additional Lidl store is on average associated with 10\% lower housing prices in the MSOA. Waitrose and Iceland are statistically significant at the 10\% level, with coefficients of 0.052 and -0.056. Despite the premium reported by \citet{LloydsBank2016LivingHome} for Iceland at £20,034 that is higher than Coop (£17,904) and Morrisons (£10,558), this negative coefficient indicates that Iceland might be closer to the lower end of the distribution. This is sensible considering that Iceland specialises in frozen food products sold at a discount, which makes their price levels more comparable to the discount stores. Marks and Spencer, Asda, Aldi and Coop have point estimates similar to aggregate levels, with limited levels of statistical significance with p-values ranging between 10\% and 20\%. Adjusted-$R^2$ for this model is identical to model (7) at 0.671, indicating that the model's explanatory power is not necessarily undermined by variable aggregation. 

\subsection{Causal Analysis of Supermarket Amenities} \label{subsection:iv}
One of the key shortcomings of cross-sectional analysis is that it only provides correlational evidence. As housing prices are dynamically determined as part of a complex urban structure, simultaneity becomes a major issue in inferring directions of causality. The cross-sectional OLS result in the previous section identified a modest premium of 5\% on housing prices for a high-end supermarket store, while low-end supermarkets are associated with a price discount of 8\%. In this section, I instrument these two variables \texttt{supermarkets\_high} and \texttt{supermarkets\_low}, and discuss the causal links between supermarket amenities and housing prices by running two-stage least squares regressions. First, I provide intuition and construction methods of two instruments, \texttt{planning} and \texttt{retail\_size\_mean}. Relevance and exclusion conditions of these instruments are examined using weak instrument test and Wu-Hausman test. IV regressions are then run for three different specifications, instrumenting one supermarket category at a time and then both high-end and low-end supermarkets.\\\\
One common characteristic among low-end supermarkets is that their establishments tend to have relatively large floor spaces, which often come with parking spaces in their suburban locations. This is an indirect consequence of zero waste policies adopted by German retailers including Lidl and Aldi, which display their products in delivery cartons without using additional wrapping or boxing. \citet{LidlUK2018IntroduceExpand}, for instance, has a site requirement policy for land developers and leasing firms, in which they specify "unit sizes flexible on design and scale between 10,000 and 25,000 sq ft" for their new locations in the Greater London area. This is considerably larger than a typical Tesco Express store that is limited to 3,000 square feet in size, and its minimum lot size of 10,000 square feet corresponds to the typical size of a Tesco Metro store that is in between 7,000 and 15,000 square feet \citep{Vasquez-Nicholson2016UK2013}. Based on this observation, I construct a variable \texttt{retail\_size\_mean} from OpenStreetMap that approximates the mean lot size of retailers in each MSOA. This value is calculated based on the geometry of polygons which have \texttt{retail} under the \texttt{building} key in map objects. Retail store sizes are expected to be positively correlated to the number of low-end stores, while the error term $\epsilon$ of model (7) in OLS regressions should be uncorrelated conditional on the other variables such as distance.\\\\
Partially related to the store structures described above, I instrument high-end supermarkets by constructing a variable that approximates planning leniency at the borough level. The basic rationale is that the high-end supermarkets are inclined to be scattered around central areas with good pedestrian access and a relatively high number of establishments. In densely populated London, this usually entails renovation of existing buildings or land redevelopment. For instance, \citet{Hawkes2010BigDay} documents that Tesco and Sainsbury's are the top two supermarket chains acquiring planning approvals from local authorities in the UK, with 392 and 111 stores accepted over the course of two years up to 2010. To establish a measure for planning leniency, I utilise planning permissions data available from \citet{GreaterLondonAuthority2017PlanningDatabase} via London Development Database, and count the number of permissions issued by each borough from fiscal years 2014 to 2018. Areas with a higher number of planning permissions are expected to have more small-scale, high-end stores such as Waitrose and Marks and Spencer. Since planning permissions are rather bureaucratically issued by local authorities, this is expected to be exogenous to the housing prices.

% Table created by stargazer v.5.2.2 by Marek Hlavac, Harvard University. E-mail: hlavac at fas.harvard.edu
% Date and time: Tue, Apr 02, 2019 - 02:19:44
\begin{table}[t] \centering 
  \caption{Instrumental Variables} 
  \label{table:iv} 
\begin{tabular}{@{\extracolsep{5pt}}lcccc} 
\\[-1.8ex]\hline 
\hline \\[-1.8ex] 
 & \multicolumn{4}{c}{\textit{Dependent variable:}} \\ 
\cline{2-5} 
\\[-1.8ex] & \multicolumn{4}{c}{log\_price} \\ 
\\[-1.8ex] & \textit{OLS} & \multicolumn{3}{c}{\textit{IV}} \\ 
\\[-1.8ex] & (1) & (2) & (3) & (4)\\ 
\hline \\[-1.8ex] 
 dist\_miles\_log & $-$0.440$^{***}$ & $-$0.348$^{***}$ & $-$0.434$^{***}$ & $-$0.354$^{***}$ \\ 
  & (0.015) & (0.053) & (0.018) & (0.048) \\ 
  & & & & \\ 
 supermarkets\_high & 0.050$^{***}$ & \textbf{1.005$^{**}$} & 0.068$^{**}$ & \textbf{0.848$^{**}$} \\ 
  & (0.018) & (0.464) & (0.032) & (0.375) \\ 
  & & & & \\ 
 supermarkets\_middle & $-$0.023$^{*}$ & $-$0.090$^{**}$ & $-$0.007 & $-$0.041 \\ 
  & (0.012) & (0.041) & (0.025) & (0.033) \\ 
  & & & & \\ 
 supermarkets\_low & $-$0.088$^{***}$ & $-$0.200$^{***}$ & \textbf{$-$0.304} & \textbf{$-$0.643$^{**}$} \\ 
  & (0.022) & (0.069) & (0.304) & (0.323) \\ 
  & & & & \\
 Constant & 14.143$^{***}$ & 13.852$^{***}$ & 14.149$^{***}$ & 14.050$^{***}$ \\ 
  & (0.037) & (0.159) & (0.040) & (0.119) \\ 
  & & & & \\ 
 Instrument: planning leniency & - & YES &  & YES \\
 Instrument: retail area size & - &  & YES & YES \\ 
  & & & & \\
\hline \\[-1.8ex] 
Weak instruments (supermarkets\_high) & - & 0.0171 &  & 0.0011  \\ 
Weak instruments (supermarkets\_low) & - &  & 0.0227 & 0.0134 \\ 
Wu-Hausman & - & 0.0000 & 0.1546 & 0.0002\\ 
Observations & 951 & 951 & 951 & 951 \\ 
Adjusted R$^{2}$ & 0.671 & - & - & - \\ 
Residual Std. Error (df = 926) & 0.258 & 0.511 & 0.272 & 0.590 \\ 
\hline 
\hline \\[-1.8ex] 
\textit{Note: IV estimates in bold}  & \multicolumn{4}{r}{$^{*}$p$<$0.1; $^{**}$p$<$0.05; $^{***}$p$<$0.01} \\ 
\end{tabular} 
\end{table}

Table \ref{table:iv} shows three specifications of IV models focusing on supermarkets. Model (1) corresponds to model (7) in OLS, with all control variables included across all specifications. Adjusted-$R{^2}$ is omitted from IV as it does not give natural interpretations, and robust standard errors are reported in brackets. Models (2) and (3) instrument high-end and low-end supermarkets individually, and model (4) uses both instruments to infer IV estimates for both variables at the same time. For relevance, weak instrument test is conducted against the null that the instrument has a weak first stage, and p-values for rejecting this null are reported in the table. In models (2) and (3), the relevance condition seems to hold for both instruments with the null hypothesis of weak instruments being rejected at the 5\% level, though exclusion condition cannot be rejected for instrument \texttt{retail\_store\_size} with p-value of 0.1546 for the  Wu-Hausman test. Nonetheless, in model (4) where both \texttt{supermarkets\_high} and \texttt{supermarkets\_low} are instrumented, these instruments are both relevant and exogenous based on these tests. This is understandable in that planning leniency and retail area size partially explain the variation in other supermarket amenities as part of the first stage, where all instruments are used for each endogenous variable.\\\\
For high-end supermarkets, models (2) and (4) both find a positive coefficient that is statistically significant at the 5\% level. However, while the signs of these coefficients are consistent with the OLS estimates, the magnitudes are considerably larger which undermines the interpretability of these models in probabilistic terms. This discrepancy potentially stems from the discrete nature of supermarket distributions. For instance, given that there are only 124 Waitrose stores across 951 MSOAs, the actual value each MSOA takes is often binary except for extremely concentrated areas. This is problematic with continuous instruments, as the first stage regression is essentially an OLS regression that is not suitable for this type of distribution that is binomial for most cases. Considering this, a more suitable experimental setup is proposed in Section \ref{subsection:limitations}. Regardless, despite the limited evidence in terms of the effect size, IV results appear to show that housing premium appears to be caused by the presence of high-end supermarkets.\\\\
On the other hand, low-end supermarket amenity appears to show a mixed result in terms of causality. While \texttt{supermarket\_low} in model (4) has a strong negative coefficient statistically significant at the 5\% level, it fails to show a consistent negative effect in model (3) with a p-value of 0.3175. This difference potentially sheds light into the issue of reverse causality for low-end supermarkets. Although it is not inconceivable that the presence of large scale discount stores entail negative externalities including pollution and activities associated with lower socioeconomic status such as crime, a supermarket itself is a positive amenity that, for instance, property agents can use for advertisement. To further investigate the endogenous determination of housing prices, a more specific approach should be employed based on individual store locations which is suggested in Section \ref{subsection:limitations}.
    
\section{Discussion} \label{section:discussion}
\subsection{External Validity}
This paper has centrally utilised geographic information data from OpenStreetMap in constructing local amenity variables. This approach itself is highly transferable across modern metropolitan areas given the map's worldwide availability, while the granularity of mapping varies largely depends on the number of volunteers in a given city. Simple variable construction method of counting entities in a given area has proven to be successful in approximating neighbourhood characteristics, where control variables such as tourist amenities and education accessibility provided comparable estimates to those in the literature. Nonetheless, the implications of these estimates should be taken with caution due to several implicit assumptions I made in the empirical process. First, these estimates are likely different in size depending on the scale of a geographic unit to represent housing prices. Whereas this paper has used MSOAs given the convenience of homogeneity in population size,  \citet{Pope2015WhenAlways} use miles radius around a Walmart store to approximate shopping areas, and \citet{Biagi2015DoesItaly} and \citet{DepartmentforEducation2017HouseMore} both use postal codes for the neighbourhood characteristics. An arbitrary choice of neighbourhood size results in limited representation of local amenities and their effect on housing prices, especially when an amenity such as a hospital or a supermarket has an influence beyond its area. This might cause spatial spillovers where a neighbouring area might gain  a pricepremium in housing prices without the explicit presence of an amenity it benefits from.\\\\
Additionally, while the distance to the CBD is integral in explaining the variation in housing prices in hedonic regression models, this is obviously subject to the assumption of monocentricity and the \textit{a priori} location of the CBD. Although this might not be applicable to polycentric cities such as Los Angeles and Seoul \citep{Park2011SpatialAngeles}, the validity of empirical results heavily relies on the structure of the extended urban area and the relative scale of target areas included in the analysis. For instance, while it is likely the case that London still exhibits some degree of polycentricity within the central area such as Zone 1, the large scope of Greater London that consist of 33 local government districts allows the distance measure to have a high explanatory power as the difference within the inner city becomes somewhat negligible for the majority of other areas outside the central area. Hence, the use and interpretation of distance measures should be approached with care considering the nature and size of the urban area to be investigated.

Lastly, instruments I developed in Section \ref{subsection:iv} might not be suitable in other contexts such as urban areas in the United States, in that the main justifications of these instruments are dependent on scarcity of land and planning restrictions. This might not translate directly to areas with different regulatory environments where planning variation does not come from the district level, or where store sizes are primarily determined by other factors as they face less scarcity of land due to the sprawling urban space, for instance.

\subsection{Limitations and Future Work}  \label{subsection:limitations}
One major drawback of the hedonic pricing approach that I constructed using average housing prices in MSOAs is that it potentially disregards the subtlety of pricing mechanisms and heterogeneity within a given area, as no consideration is given for individual housing characteristics or demographic information. While this was effective in understanding the overall picture of housing price distribution in Greater London and some underlying characteristics in the form of local amenities, it remains problematic to provide causal effects of specific entities such as supermarkets. To investigate further into their causal effects, the difference-in-difference estimation adopted by \citet{Pope2015WhenAlways} and \citet{Slade2016Walmart} in their studies of Walmart openings appears more suitable as they pay particular attention to surrounding factors in examining the changes in housing prices before and after the announcement and opening of a store. While the historical data for London is only available from 2014 for OpenStreetMap, some control variables such as land use, nature and school can be used as fixed effects. \citet{Slade2016Walmart} demonstrates an intriguing finding that the land prices start to increase in the surrounding areas during the development phase of a new Walmart store. This gives insight into different timings when the premium comes into effect. The data source for housing prices, land registry, provides records on individual sales of properties, and a more granular approach is feasible while utilising the methods developed in this paper. Additionally, some amenities might be better represented with land shares in a given area when there is a large variance within the same tag, such as parks and playgrounds as highlighted in Section \ref{subsection:variable}. This information can be extracted from the same map objects used in this paper and potentially provides better controls for nature amenities. 

\section{Conclusion} \label{section:conclusion}
Applying the monocentric city model \citep{AlonsoWilliam1964Lalu} and hedonic pricing model \citep{Rosen1984}, I present a comprehensive price structure of the housing market in London through neighbourhood amenities. Under this framework, distance to the CBD is found to be the predominant factor for housing prices. Morning commute time approximated by Uber Movement is insightful as it shows a steeper discount to housing prices, where a 1\% increase in travel times is associated with a 0.69\% increase in housing prices. This implies that consumers are more concerned about the actual commuting time than the geometric distance to the CBD.\\\\
Crucially, I provide novelty to the literature of supermarket amenities in the housing market. High-end supermarkets such as Waitrose and Marks and Spencer are estimated to provide a housing price premium of 5\% per store, while low-end supermarkets such as Lidl and Aldi are associated with a discount of 8\%, in contrast to a popular report by \citet{LloydsBank2016LivingHome} where they provide a modest premium of 1 to 2\%. Through the use of instrumental variables, I demonstrate limited evidence on the causal directions of supermarket amenities, particularly for the high-end supermarket chains. These results are underpinned by the construction of novel spatial features in which standardised tags and availability of a vast number of map objects are utilised to approximate several local amenities. Several measures are taken to establish plausible estimates by carefully selecting control variables. Lack of specificity in the type of schools, for instance, is augmented by a compilation of independent school data through web scraping, which improved the overall explanatory power of the model by 0.04 in adjusted-$R{^2}$.The synthesis of these datasets provides an improvement to the limited evidence by \citet{LloydsBank2016LivingHome} that disregards the neighbourhood heterogeneity, as the models presented in this paper control for other amenities that could affect the housing prices.\\\\
Policy implications for causal effects of supermarkets are large. For urban planning, a local government certainly benefits from considering effects of local amenities in their cost-benefit analysis for issuing planning permissions or attracting certain types of establishments. Crucially, the negative coefficients found for low-end supermarkets in this paper could be particularly consequential in terms of property and income taxes. Negative externalities that are potentially associated with low-end supermarkets such as pollution, congestion and crime, are also policy concerns for local authorities themselves. Therefore, the magnitudes of causal effects need to be investigated further based on individual supermarket openings, considering more granular characteristics of individual housing prices in the neighbourhood using time series data.

\newpage
\nocite{*}
\renewcommand\harvardyearleft{\unskip, }
\renewcommand\harvardyearright[1]{.}
\let\oldthebibliography\thebibliography
\renewcommand\thebibliography{\let\bf\relax\oldthebibliography}
\renewcommand{\refname}{\textbf{Bibliography}}
\bibliographystyle{agsm}  
\bibliography{bibliography.bib}

% Table created by stargazer v.5.2.2 by Marek Hlavac, Harvard University. E-mail: hlavac at fas.harvard.edu
% Date and time: Sun, Mar 31, 2019 - 06:10:02

\newpage
\textbf{Appendix: Complete OLS Table}
\begin{table}[H] \centering 
  \caption{OLS: Neighbourhood Amenities} 
  \label{table:complete} 
\small 
\begin{tabular}{@{\extracolsep{-10pt}}lc} 
\\[-1.8ex]\hline 
\hline \\[-1.8ex] 
 & \multicolumn{1}{c}{\textit{Dependent variable:}} \\ 
\cline{2-2} 
\\[-1.8ex] & log\_price \\ 
\hline \\[-1.8ex] 
 dist\_miles\_log & $-$0.440$^{***}$ (0.015) \\ 
  & \\ 
 supermarkets\_high & 0.050$^{***}$ (0.018) \\ 
  & \\ 
 supermarkets\_middle & $-$0.023$^{*}$ (0.012) \\ 
  & \\ 
 supermarkets\_low & $-$0.088$^{***}$ (0.022) \\ 
  & \\ 
 residential & 0.007$^{***}$ (0.002) \\ 
  & \\ 
 retail & 0.001 (0.002) \\ 
  & \\ 
 industrial & $-$0.022$^{***}$ (0.005) \\ 
  & \\ 
 construction & 0.008 (0.005) \\ 
  & \\ 
 commercial & $-$0.018$^{***}$ (0.006) \\ 
  & \\ 
 garden & 0.0002 (0.001) \\ 
  & \\ 
 pitch & 0.003$^{**}$ (0.001) \\ 
  & \\ 
 park & $-$0.003$^{*}$ (0.002) \\ 
  & \\ 
 playground & $-$0.022$^{***}$ (0.006) \\ 
  & \\ 
 nature\_reserve & 0.035 (0.026) \\ 
  & \\ 
 golf\_course & 0.144$^{***}$ (0.033) \\ 
  & \\ 
 hotel & $-$0.004 (0.004) \\ 
  & \\ 
 attraction & 0.018$^{**}$ (0.009) \\ 
  & \\ 
 museum & 0.021 (0.028) \\ 
  & \\ 
 art & $-$0.007 (0.023) \\ 
  & \\ 
 pub\_restaurant & 0.010$^{***}$ (0.003) \\ 
  & \\ 
 cafe & 0.003 (0.005) \\ 
  & \\ 
 fast\_food & $-$0.025$^{***}$ (0.005) \\ 
  & \\ 
 private\_school & 0.124$^{***}$ (0.014) \\ 
  & \\ 
 university & $-$0.002 (0.006) \\ 
  & \\ 
 Constant & 14.140$^{***}$ (0.037) \\ 
  & \\ 
\hline \\[-1.8ex] 
Observations & 951 \\ 
R$^{2}$ & 0.680 \\ 
Adjusted R$^{2}$ & 0.671 \\ 
\hline 
\hline \\[-1.8ex] 
\textit{Note: SE in parenthesis}  & \multicolumn{1}{r}{$^{*}$p$<$0.1; $^{**}$p$<$0.05; $^{***}$p$<$0.01} \\ 
\end{tabular} 
\end{table} 

\end{document}